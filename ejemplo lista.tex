\documentclass{article}
\usepackage[utf8]{inputenc}
\usepackage{color}
\usepackage{enumerate}
\usepackage{array}
\usepackage[pdftex]{graphicx}



\begin{document}

\centerline{\Large\underline{Listas y tablas sobre el Swag}}

\bigskip

\renewcommand{\labelitemi}{\Large\color{red}\textbullet}
\addtolength{\leftmarginii}{3cm}
\begin{itemize}
 \item Otro pequeño ejemplo de cómo podemos anidar los 
 \begin{enumerate}[\bfseries Obj{e}tiv{o} 1 \hspace{3mm}]
  \item Historia del Swag
  \item Biología del Swag
  \item Swag en la actualidad
 \end{enumerate}
 \item Segundo apartado,que contiene a su vez tres 
 \begin{enumerate}[\color{blue}\itshape Secc{i}ón A \hspace{3mm}]
  \item Bioquímica del Swag
  \item Genética del Swag
  \item Epigenética del Swag
 \end{enumerate}
\end{itemize}


\begin{figure}[h]
\includegraphics[width=120mm]{Dollar.jpg}
\caption{Representación molecular y real del Swag}
\label{Suuuu}
\end{figure}

\newpage

\begin{equation}
\frac{Topor}{Keylor}=Energias Renovables \\ \
\alpha Swag \bullet \delta Flow = \oint Hype + RealSwagga
\end{equation}

\begin{equation}
\frac{Flow Oxidado}{Flow Reducido}= Expresion \\ \ genica \\ \ del \\ \ Flow \\ \ en \\ \ el \\ \ organismo
\end{equation}

\begin{equation}
\int_{Flow}^{Swag} f(hype)
\end{equation}
\end{document}